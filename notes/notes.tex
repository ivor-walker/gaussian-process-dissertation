\documentclass[12pt]{article}

\usepackage[utf8]{inputenc}

\usepackage{geometry}
\geometry{a4paper, margin=1in}

\usepackage{amsmath}
\usepackage{amssymb}

\usepackage{hyperref}

% For paragraphs
\usepackage{titlesec}
\setcounter{secnumdepth}{5}

\usepackage[
    backend=biber,
    style=numeric,
    url=true
]{biblatex}
\addbibresource{../../references/references.bib}

\begin{document}




\section{Regression (\cite{gp-ml} Chapter 2)}



\subsection{Weight-space view}


\subsubsection{Standard linear model}

\paragraph{Standard and Bayesian model definitions}
\begin{itemize}
    \item We're trying to learn the distribution $p(y|X,W)$
    \begin{itemize}
        \item $X$ is the input data, $W$ is the model parameters, $y$ is the output
        \item $p(y|X,W)$ is the conditional distribution of $y$ after everything we know about $X$ and $W$, distribution of errors
    \end{itemize}
    \item Standard linear model: $f(X) = X^T W, y = f(X) + \epsilon$ with our Gaussian noise $\epsilon \sim \mathcal{N}(0, \sigma^2_nI)$, which produces $p(y|X,W) = \mathcal{N}(y|f(X), \sigma^2_n)$
    \item Bayesian linear model: firstly, specify the linear prior distribution $p(W)$ over the weights $p(W) \sim \mathcal{N}(0, \Sigma_p)$
    \begin{itemize}
        \item A prior $p(W)$ expresses our beliefs about the parameters before we see the data
        \item Linear model specifies that our weights follow a zero mean Gaussian prior with a covariance matric $\Sigma_p$
    \end{itemize}
    \item Then, we update our beliefs about the weights after seeing the data, using Bayes' theorem
\begin{equation}
    \text{posterior} = \frac{\text{likelihood} \times \text{prior}}{\text{marginal likelihood}}
\end{equation}
\begin{equation}
    p(W|X, y) = \frac{p(y|X,W)p(W)}{p(y|X)}
\end{equation}
    \begin{itemize}
        \item $p(y|X,W)$ is the density of the residuals after applying our priors $p(W)$ to the data $X,W$ under our assumed noise model $\epsilon$
        \item $p(W)$ is the prior distribution of the weights
        \item $p(y|X)$ is the marginal likelihood, which is the probability of the data given the model
\begin{equation}
    p(y|X) = \int p(y|X,W)p(W)dW
\end{equation}
        \item $p(y|X)$ is the normalising constant, ensures the posterior distribution integrates to 1
        \item $p(W|x, y)$ is the distribution of the weights given the data - combines the likelihood and the prior, representing everything we know about the parameters
    \end{itemize}
    \item To understand how our posterior varies with our weights, we can write terms that only depend on weights (i.e. likelihood and prior, not marginal likelihood)
\begin{equation}
    p(W|X, y) \propto p(y|X,W)p(W)
\end{equation}
     \begin{itemize}
         \item We will adopt the same idea throughout: if a term doesn't depend on weights, we simply remove it
     \end{itemize}
\end{itemize}

\paragraph{Deriving our posterior}
\begin{itemize}
     \item Given our linear model $f(X) = X^TW$ and our Gaussian noise $\epsilon$, we can write $p(Y|X,W)$ as the distribution of errors for each data point $i$
 \begin{equation}
     p(y|X,W) = \prod_{i=1}^N \mathcal{N}(y_i|X^TW, \sigma^2_n)
 \end{equation}
     \item We can find $p(y|X,W)$ by multiplying the Gaussian density $-\frac{1}{2\sigma^2_n}$ and the squared errors from our model $||y -X^TW||^2$, so our final likelihood becomes
 \begin{equation}
     p(y|X,W) = exp\left(-\frac{1}{2\sigma^2_n}||y -X^TW||^2\right)
 \end{equation}
     \item Given that our $p(W)$ is a zero mean Gaussian prior with covariance $\Sigma_p$, we can substitute this into the Gaussian density function
 \begin{equation}
     p(W) = \frac{1}{[\sqrt{\sigma_p}]\sqrt{2\pi}} exp\left(-\frac{1}{2}\frac{([W]-[0])}{[\Sigma_p]}\right)
 \end{equation}
     \item The first term of this $p(W)$ is another normalising constant, so rewriting the fraction in the exponent as a negative exponential gives us
 \begin{equation}
     p(W) \propto exp\left(-\frac{1}{2}W^T\Sigma_p^{-1}W\right)
 \end{equation}
     \item Putting both expressions for $p(y|X,W)$ and $p(W)$ together, we can write the posterior as
 \begin{equation}
     p(W|X,y) \propto exp\left(-\frac{1}{2\sigma^2_n}||y -X^TW||^2\right)exp\left(-\frac{1}{2}W^T\Sigma_p^{-1}W\right)
 \end{equation}
     \item To simplify, first we can expand $||y - X^TW||^2$ to $y^Ty - 2y^TXW + W^TX^TXW$, and substitute this expanded expression to get
 \begin{equation}
     p(W|X,y) \propto exp\left(-\frac{1}{2\sigma^2_n}(y^Ty - 2y^TXW + W^TX^TXW)\right)exp\left(-\frac{1}{2}W^T\Sigma_p^{-1}W\right)
 \end{equation}
     \item Then, we put both exponentials together (by adding their powers)
 \begin{equation}
     p(W|X,y) \propto exp\left(\frac{1}{\sigma^2_n}(y^Ty - 2y^TXW + W^TX^TXW) + \left(-\frac{1}{2}W^T\Sigma_p^{-1}W\right)\right)
 \end{equation}
     \item We can rearrange the inside term to be a quadratic, linear and constant term in $W$:
 \begin{equation}
     p(W|X,y) \propto exp\left(\frac{1}{2}W^T\left(\frac{1}{\sigma^2_n}X^TX + \Sigma_p^{-1}\right)W - \left(\frac{1}{\sigma^2_n}y^TX\right)W + \frac{1}{2}y^Ty\right)
 \end{equation}
     \item We can ignore the constant final term, and introduce $A = \Sigma_p^{-1} + \frac{1}{\sigma^2_n}X^TX$ and $b = \frac{1}{\sigma^2_n}y^TX$ to get
 \begin{equation}
     p(W|X,y) \propto exp\left(-\frac{1}{2}W^TAW + b^TW\right)
 \end{equation}
\end{itemize}

\paragraph{Deriving the properties of the posterior by completing the square}
\begin{itemize}
     \item Now we have a simplified form of the posterior density, we need to get it into a Gaussian form to recover the properties of the posterior distribution
    \item Firstly, we can bring all terms inside the exponential to a single term
\begin{equation}
    -\frac{1}{2}W^TAW + b^TW = \frac{1}{2}\left(-W^TAW + 2b^TW\right)
\end{equation}
    \item We can "complete the square" on this term $W^TAW - 2b^TW$ to rewrite it in a form that is easier to interpret
\begin{equation}
    W^TAW - 2b^TW = (W - A^{-1}b)^TA(W - A^{-1}b) - b^TA^{-1}b
\end{equation}
    \item Substituting this back into our posterior density gives us
\begin{equation}
    p(W|X,y) \propto exp\left(-\frac{1}{2}\left((W - A^{-1}b)^TA(W - A^{-1}b) - b^TA^{-1}b\right)\right)
\end{equation}
    \item If we look at our Gaussian density: 
\begin{equation}
    N(W | \mu, \Sigma) = \frac{1}{\sqrt{(2\pi)^d |\Sigma|}} exp\left(-\frac{1}{2}(W - \mu)^T\Sigma^{-1}(W - \mu)\right)
\end{equation}
    \item We can see that our expression lines up with the RHS Gaussian "kernel" term $exp\left(-\frac{1}{2}(W - \mu)^T\Sigma^{-1}(W - \mu)\right)$, where $\mu = A^{-1}b$ and $\Sigma^{-1} = A$ thus $\Sigma = A^{-1}$
    \item So we can write our posterior density in Gaussian form
\begin{equation}
    p(W|X,y) \sim N(A^{-1}b, A^{-1})
\end{equation}
\end{itemize}

\paragraph{Gaussian posteriors and ridge regression}
\begin{itemize}
    \item For Gaussian posteriors, our mean $A^{-1}b$ is also its mode, called the maximum a posteriori (MAP) estimate of W
        \begin{itemize}
            \item Due to symmetries in linear model and posterior, not the case in general
        \end{itemize}
    \item In non-Bayesian settings the MAP point is the MLE estimation
    \item $B$ is dependent on $\sigma_n^2$, $y$ and $X$ - all known
    \item $A$ is dependent on $\sigma_n^2$, $X$ and $\Sigma_p$ - all known except $\Sigma_p$
    \item Our weight variance $\Sigma_p$ under the Bayesian linear model is "isotropic", meaning it is the same in all directions
\begin{equation}
    \Sigma_p = \tau^2 I
\end{equation}
    \begin{itemize}
        \item $I$ is our $D \times D$ correlation matrix, here we assume independence so our correlation matrix is an "identity matrix" (each diagonal element is 1 and all off-diagonal elements are 0)
        \item $\tau^2$ is a scalar variance term, chosen as a prior
    \end{itemize}
    \item We can substitute our new isotropic prior $\Sigma_p$ into $A$ to get
\begin{equation} 
    A = \Sigma_p^{-1} + \frac{1}{\sigma^2_n}X^TX = \left[{\tau^2}I\right]^{-1} + \frac{1}{\sigma^2_n}X^TX =  \frac{1}{\tau^2}I + \frac{1}{\sigma^2_n}X^TX = \frac{1}{\sigma_n^2}\left(X^TX + \frac{\sigma_n^2}{\tau^2}I\right)
\end{equation} 
    \item Now we have full expressions for $A$ and $B$, we can substitute them into our MAP estimation for W to get
\begin{equation} 
    W_{\text{MAP}} = A^{-1}b = \left[\frac{1}{\sigma_n^2}(X^TX + \frac{\sigma_n^2}{\tau^2}I\right]^{-1} \cdot \left[\frac{1}{\sigma_n^2}y^TX\right] \\
\end{equation}
    \item We can compute the LHS inversion of $A$
\begin{equation}
    A^{-1} = \frac{\sigma_n^2}{X^TX + \frac{\sigma_n^2}{\tau^2}I} = \sigma_n^2\left(X^TX + \frac{\sigma_n^2}{\tau^2}I\right)^{-1}
\end{equation}
    \item Substituting this back into $W_\text{MAP}$ cancels out the $\sigma_n^2$ term in A with the $\frac{1}{\sigma_n^2}$ term in B, giving us
\begin{equation}
    W_{\text{MAP}} = \sigma_n^2\left(X^TX + \frac{\sigma_n^2}{\tau^2}I\right)^{-1} \cdot \frac{1}{\sigma_n^2}y^TX = \left(X^TX + \frac{\sigma_n^2}{\tau^2}I\right)^{-1} \cdot y^TX
\end{equation}
    \item The solution to ridge regression is very similar
\begin{equation}
    W_{\text{ridge}} = \left(X^TX + \lambda I\right)^{-1}X^Ty
\end{equation}
    \item In ridge regression, $\lambda$ is a regularisation parameter that controls the amount of shrinkage which is usually selected to maximise likelihood/minimise error
    \item MAP estimation in Bayesian linear regression with isotropic priors is equivalent to ridge regression with a regularisation parameter $\lambda = \frac{\sigma_n^2}{\tau^2}$
        \begin{itemize}
            \item The higher our $\lambda$, the more biased our model is towards the prior, and the more we shrink our weights towards zero and the prior has more influence
            \item A lower $\tau$ causes a higher $\lambda$ - smaller weight variances around zero means lower weights because of higher confidence in priors
            \item A higher $\sigma_n$ also causes a higher $\lambda$ - larger noise variances means lower weights because of lower confidence in weights forces deference to the prior
        \end{itemize}
\end{itemize}

\paragraph{Deriving the predictive distribution}
\begin{itemize}
    \item Ultimately, our goal is to approximate a data-generating function $f_*$ (or a new observation $y_*$) that produced a new $X_*$ given training data $X$ and $y$ and weights $W$ from the same $f_*$  
    \item In non-Bayesian frameworks, we make predictions by choosing a single parameter value $W$ to maximise the likelihood of the data, which is our MLE estimate
    \item In a Bayesian framework, we average over all possible parameter values weighted by their posterior probability $p(W|X,y)$, e.g. for a linear model $\hat{W} = \mathbb{E}_{p(W|X,y)}[W] = W_\text{MAP} = A^{-1}b$
    \item In this framework, we can make comments about our uncertainty of $W$ by forming a "predictive distribution" $p(f_*|X_*,X,y)$
\begin{equation}
    p(f_*|X_*,X,y) = \int p(f_*|X_*,W) \cdot p(W | X,y)dW
\end{equation}
    \begin{itemize}
        \item $p(f_*|X_*,W)$ is what we think the function looks like after producing a prediction using $X_*$ and perfect knowledge of $W$
        \item $p(W|X,y)$ is the posterior distribution of the weights given the training data, e.g. minimised for $W_\text{MAP}$
        \item $p(f_*|X_*,W) \cdot p(W|X,y)$ is the joint distribution of our predictions and our posterior weights, which gets us the conditional distribution $p(f_*,W|X_*,X,y)$ by definition of conditional probability
        \item $p(f_*,W|X_*,X,y)$ relies on our perfect knowledge of $W$, which we don't have, so we integrate over all possible $W$ to get the predictive distribution $p(f_*|X_*,X,y)$
    \end{itemize}
    \item We already know $p(W|X,y)$
\begin{equation}
    p(W|X,y) \propto exp\left(\frac{1}{2}(-W^TAW + 2b^TW)\right)
\end{equation}
    \item $p(f_*|X_*,W)$ is our errors, which we assume to be distributed normally and independently with our $I$ identity matrix:
\begin{equation}
    p(f_* | X_*, W) = \mathcal{N}(f_* | W^TX_*, \sigma^2_nI)
\end{equation}
    \item Plugging these into our Gaussian density and ignoring the LHS normalisation term yields
\begin{equation}
    p(f_*|X_*,w) \propto exp\left(-\frac{1}{2}\frac{1}{\sigma^2_n}(f_* - W^TX_*)^2\right)
\end{equation}
    \item We can multiply $P(f_*|X_*,W)$ and $p(W|X,y)$ to get our conditional $p(f_*, W|X_*,X,y)$, and add the exponents to simplify
\begin{equation}
    p(f_*,W|X_*,X,y) \propto exp\left(\frac{1}{2}(-W^TAW + 2b^TW) + \left(-\frac{1}{2}\frac{1}{\sigma^2_n}(f_* - W^TX_*)^2\right)\right)
\end{equation}
    \item We can further combine these with a single factor of $\frac{1}{2}$ to get
\begin{equation}
    p(f_*,W|X_*,X,y) \propto exp\left(-\frac{1}{2}\left(W^TAW - 2b^TW + \frac{1}{\sigma^2_n}(f_* - W^TX_*)^2\right)\right)
\end{equation}
    \item Expanding the squared term gives us
\begin{equation}
    p(f_*,W|X_*,X,y) \propto exp\left(-\frac{1}{2}\left(W^TAW - 2b^TW + \frac{1}{\sigma^2_n}(f_*^2 - 2f_*W^TX_* + W^TX_*X_*^TX_*)\right)\right)
\end{equation}
    \item Similar to our posterior, we can rearrange this to be a quadratic, linear and constant term in $W$
\begin{equation}
    p(f_*,W|X_*,X,y) \propto exp\left(-\frac{1}{2}\left(W^T\left(A + \frac{1}{\sigma^2_n}X_*X_*^T\right)W - 2\left(b + \frac{1}{\sigma^2_n}f_*X_*\right)^TW + \frac{1}{\sigma_n^2}f_*^2\right)\right)
\end{equation}
    \item By defining $A_* = A + \frac{1}{\sigma^2_n}X_*X_*^T$ and $b_* = b + \frac{1}{\sigma^2_n}f_*X_*$, we can rewrite this as
\begin{equation}
    p(f_*,W|X_*,X,y) \propto exp\left(-\frac{1}{2}\left(W^TA_*W - 2b_*^TW + \frac{1}{\sigma_n^2}f_*^2\right)\right)
\end{equation}
    \item We have to integrate this wrt $W$ to get our predictive distribution $p(f_*|X_*,X,y)$
\begin{equation}
    p(f_*|X_*,X,y) = \int p(f_*,W|X_*,X,y)dW \propto \int exp\left(-\frac{1}{2}\left(W^TA_*W - 2b_*^TW + \frac{1}{\sigma_n^2}f_*^2\right)\right)dW
\end{equation}
    \item We can factor out the $\frac{1}{\sigma_n^2}f_*^2$ term from the integral, as it does not depend on $W$ so remains the same since $\int exp(X) dX = exp(X)$ 
\begin{equation}
    = exp\left(-\frac{1}{2}\frac{1}{\sigma_n^2}f_*^2\right) \times \int exp\left(-\frac{1}{2}\left(W^TA_*W - 2b_*^TW\right)\right) dW
\end{equation}
    \item The RHS term is a multivariate Gaussian integral (beyond your paygrade) which evaluates to:
\begin{equation}
    \int exp\left(-\frac{1}{2} \left( W^TA_*W - 2b_*^TW \right) \right) dW = \frac{(2\pi)^{D/2}} {\sqrt{|A_*|}} exp\left( \frac{1}{2} b_*^TA_*^{-1}b_* \right)
\end{equation}
    \item Substituting this back into our predictive distribution gets us
\begin{equation}
    p(f_*|X_*,X,y) \propto exp\left(-\frac{1}{2}\frac{1}{\sigma_n^2}f_*^2\right) + \frac{(2\pi)^{D/2}}{\sqrt{|A_*|}} \cdot exp\left(\frac{1}{2}b_*^TA_*^{-1}b_*\right)  
\end{equation}
    \begin{itemize}
        \item Note that no part of our expression is now dependent on W
    \end{itemize}
    \item Now we need an expression of everything that changes $f_*$
    \item Absorb the second term, since it does not depend on $f_*$ into the proportionality constant, and combining the remaining exponential terms by adding their powers gives us
\begin{equation}
    p(f_*|X_*,X,y) \propto \exp\left(-\frac{1}{2}\frac{1}{\sigma_n^2}f_*^2 + \frac{1}{2}b_*^TA_*^{-1}b_*\right)
\end{equation}
    \item Similar to deriving properties from our posterior, we can rearrange this expression, complete the square and derive the properties of our predictive distribution
\begin{equation}
    p(f_*|X_*,W) \sim N(X_*^TA^{-1}b, X_*^TA^{-1}X_*)
\end{equation}
    \item Note that the variance should have $+ \sigma_n^2$ if dealing with $y_*$
    \item Predictive variance is quadratic form of test input with $A^{-1}$, showing that predictive uncertainties grow with size of $X_*$
\end{itemize}


\subsubsection{Projections of inputs into feature space}
\begin{itemize}
    \item Bayesian linear models suffer from limited expressiveness due to the linearity of the model
    \item To address this, we can project our inputs into a higher dimensional feature space and apply linear model in this space
    \item e.g. a scalar $x$ could be projected into the space of powers of $x$: $\phi(x) = [1, x, x^2, \ldots, x^d]^T$ for a polynomial basis expansion of degree $d$
    \item How to choose $\phi(x)$? Gaussian process formalism allows us to answer this question, but for now assume $\phi(x)$ is a given
    \item $\phi(X)$ maps a D-dimensional input vector $X$ into an $N$ dimensional feature space
    \item So our full model looks like:
\begin{equation}
    f(X) = \phi(X)^T W
\end{equation}
%   \item $\phi(x)$ must be independent of $W$ so that we can learn $W$ from the data
    \item And our predictive distribution becomes
\begin{equation}
    p(f_*|X_*,X,y) = N(\phi(X_*)^TA_{\phi}^{-1}b_{\phi} , \phi(X_*)^TA_{\phi}^{-1}\phi(X_*))
\end{equation}
    \begin{itemize}
        \item $A_{\phi} = \Sigma_p^{-1} + \frac{1}{\sigma^2_n}\phi(X)^T\phi(X)$
        \item $b_{\phi} = \frac{1}{\sigma^2_n}\phi(X)^Ty$
    \end{itemize}
\end{itemize}


\subsubsection{Computational issues}

\paragraph{Avoiding inversion of $A_{\phi}$}
\begin{itemize}
    \item This formulation of our predictive distribution inverts the $N \times N$ matrix $A_{\phi}$, where $N$ is dimension of feature space, to get the expected value and variance
    \item Inverting matrices is $O(N^3)$ - not feasible for large $N$ - so we need to restate our predictive distribution in a form that avoids this inversion
    \item Substitute $b_{\phi}$ into our predictive distribution mean
\begin{equation}
    \mathbb{E}_{p(f_*|X_*,X,y)}[f_*] = \phi(X_*)^T \cdot A_\phi^{-1} \cdot \left[\frac{1}{\sigma_n^2}\phi(X)^Ty\right]
\end{equation}
    \item Rearranging to isolate $A_\phi^{-1}\phi(X)$
\begin{equation}
    = \frac{1}{\sigma_n^2}\left[A_{\phi}^{-1}\phi(X)\right]^Ty
\end{equation}
    \item We can use the Sherman-Morrison identity (beyond your paygrade) to get an expression for $A_{\phi}^{-1}$ directly, where $K =\phi(X)^T\Sigma_p\phi(X)$
\begin{equation}
    A_{\phi}^{-1} = \Sigma_p - \Sigma_p\phi(X)(K+\sigma_n^2I)^{-1}\phi(X)^T\Sigma_p
\end{equation}
    \item For the mean, we can use the Sherman-Morrison identity again to get an expression for $A_{\phi}^{-1}\phi(X)$
\begin{equation}
    A_{\phi}^{-1}\phi(X) = \sigma_n^2\Sigma_p\phi(X)(K+\sigma_n^2I)^{-1}
\end{equation}
    \item Substitute in this expression for $A_{\phi}^{-1}\phi(X)$ into our predictive distribution mean
\begin{equation}
    \mathbb{E}_{p(f_*|X_*,X,y)}[f_*] = \phi(X_*) \frac{1}{\sigma_n^2}\left[\sigma_n^2\Sigma_p\phi(X)(K+\sigma_n^2I)^{-1}\right]^Ty
\end{equation}
    \item $\frac{1}{\sigma_n^2}$ and $\sigma_n^2$ cancel out, leaving us with this final expression for the mean
\begin{equation}
    \mathbb{E}_{p(f_*|X_*,X,y)}[f_*] = \phi(X_*)^T \cdot \Sigma_p\phi(X)(K+\sigma_n^2I)^{-1}y
\end{equation}
    \item For the variance, we can't use the Sherman-Morrison identity to arrive at an expression for $A_{\phi}^{-1}\phi(X_*)$ because $\phi(X_*)$ is an arbitrary N-vector, not one of the columns of $\phi(X)$
    \item Instead, we use the $A_{\phi}^{-1}$ expression we derived earlier to get an expression for $A_{\phi}^{-1}\phi(X_*)$
\begin{equation}
    A_{\phi}^{-1}\phi(X_*) = \Sigma_p \cdot \phi(X_*) - \Sigma_p\phi(X)(K+\sigma_n^2I)^{-1}\phi(X)^T\Sigma_p \cdot \phi(X_*)
\end{equation}
    \item Substituting this into our predictive distribution variance gives us this final expression
\begin{equation}
    \text{Var}_{p(f_*|X_*,X,y)}[f_*] = \phi(X_*)^T\Sigma_p\phi(X_*) - \phi(X_*)^T\Sigma_p\phi(X)(K+\sigma_n^2I)^{-1}\phi(X)^T\Sigma_p\phi(X_*)
\end{equation}
    \item So our final predictive distribution is
\begin{equation*}
    \begin{aligned}
        p(f_*|X_*,X,y) = \mathcal{N}( \\
        \phi(X_*)^T\Sigma_p\phi(X)(K+\sigma_n^2I)^{-1}y , \\
        \phi(X_*)^T\Sigma_p\phi(X_*) - \phi(X_*)^T\Sigma_p\phi(X)(K+\sigma_n^2I)^{-1}\phi(X)^T\Sigma_p\phi(X_*) \\
        )
    \end{aligned}
\end{equation*}
    \item With this mean and variance, we need to invert $n \times n$ matrix $K + \sigma_n^2I$, where $n$ is the number of training data points, instead of $N \times N$ matrix $A_{\phi}$, where $N$ is the dimension of the feature space
    \item This formulation is faster if $n < N$
        \begin{itemize}
            \item For polynomial basis expansions, $N$ is degree $D$ multiplied by number of features, so $N$ can be very large
            \item Some kernels (e.g. RBF) have infinite dimensional feature spaces, so $N$ is infinite
            \item Some data domains (e.g. text classification, genomic data) have very high dimensional feature spaces
        \end{itemize}
    \item Geometrically, note that $n$ datapoints can span at most $n$ dimensions in the feature space - if $N > n$, the data forms a subspace of the feature space
\end{itemize}

\paragraph{Kernels and the kernel trick}
\begin{itemize}
    \item In the alternative formulation, note that $\phi$ is always in the same general form but different combinations of $\phi(X)$ and $\phi(X_*)$, which generalises to $\phi(X)^T\Sigma_p\phi(X')$ where $X$ and $X'$ are either in training $\phi(X)$ or test $\phi(X_*)$ data
    \begin{itemize}
        \item In the first term of the mean and second term of the variance, $\phi(X_*)^T\Sigma_p\phi(X)$
        \item In the first term of the variance, $\phi(X_*)^T\Sigma_p\phi(X_*)$
        \item In the last term of the variance, $\phi(X)^T\Sigma_p\phi(X_*)$
        \item In the definition of $K = \phi(X)^T\Sigma_p\phi(X)$
    \end{itemize}
    \item We can define $k(X,X') = \phi(X)^T\Sigma_p\phi(X')$ as a covariance function or kernel
    \item Note that this kernel is an inner product with a positive definite correlation matrix $\Sigma_p$
    \item This lets us define $\psi(X) = \Sigma_p^{1/2}\phi(X)$
    \item Substituting $\psi(X)$ back into our kernel allows the $\Sigma_p^{1/2}$ to cancel out, giving us a simple dot product representation  
\begin{equation}
    k(X,X') = \psi(X) \cdot \psi(X')
\end{equation}
    \item Now we have defined a kernel solely in terms of inner products in the input space, we know that there must be another equivelant $k(X, X'$ that does not require us to explicitly compute $\phi(X)$ or $\phi(X')$ in the feature space
    \begin{itemize}
        \item e.g. if we had $\phi(X) = [1, x^1, ..., x^D]^T$ and $\Sigma_p$ as an identity matrix
        \item We could define $k(X,X')$ using $\psi$
\begin{equation}
    k(X,X') = \psi(X) \cdot \psi(X') = \phi(X) \cdot \phi(X') = [1, x^1, ..., x^D]^T \cdot [1, x'^1, ..., x'^D]^T
\end{equation}
        \item This approach requires arranging $\phi$ and $\phi(X')$ into a $D$ sized vector, then taking the dot product
        \item This is trivial for small $D$, but as $D$ becomes infinite (e.g. RBF kernel), arranging a $D$ sized vector requires too much memory
        \item Instead, we can define $k(X,X')$ as an equivelant function of $X$ and $X'$ directly
\begin{equation}
    k(X,X') = (1 + XX')^D
\end{equation}
        \item This is the polynomial kernel, which is equivalent to the polynomial basis expansion $\phi(X)$
        \item We still need to perform the same calculations but we avoid the memory cost of explicitly computing $\phi(X)$ and $\phi(X')$ in the feature space
    \end{itemize}
    \item It is mostly more convenient to compute the kernel directly rather than the feature vectors themselves, leading to the kernel being the object of primary interest
\end{itemize}



\subsection{Function-space view}


\subsubsection{Gaussian processes (GP)}

\paragraph{Definition}
\begin{itemize}
    \item A GP is a collection of random variables, any finite number of which have a joint Gaussian distribution
    \item GPs describe a distribution over functions, where each function is a sample from the GP
    \item Defined completely by its mean function $m(X)$ and covariance function $k(X,X')$ of a real process $f(X)$
\begin{equation}
    \begin{aligned}
        f(X) \sim \mathcal{GP}(m(X), k(X,X')), \\
        m(X) = \mathbb{E}[f(X)], \\
        k(X,X') = \text{Cov}(f(X), f(X')) = \mathbb{E}[(f(X) - m(X))(f(X') - m(X'))]
    \end{aligned}
\end{equation}
    \item These random variables represent the value of $f(X)$ at location $X$, e.g. often Gaussian processes are defined over time so $X$ can be a time point
    \item The covariance function specifies the covariance between pairs of random variables
\end{itemize}

\paragraph{Consistency requirement}
\begin{itemize}
    \item This definition implies a consistency requirement that means that examining a larger set does not change the distribution of a smaller set
        \begin{itemize}
            \item e.g. if GP implies that $(f(X_1), f(X_2)) \sim \mathcal{N}(\mu, \Sigma)$, then it must specify $(f(X_1) \sim \mathcal{N}(\mu_1, \Sigma), f(X_2) \sim \mathcal{N}(\mu_2, \Sigma))$ where $\mu_{\theta} = m(X_{\theta})$ and $\Sigma_{\theta\theta} = k(X_{\theta}, X_{\theta})$
        \end{itemize}
    \item This requirement is also called the marginalisation property because to get the smaller distribution of $f(X_1)$, we marginalise out the larger distribution of $f(X_1), f(X_2)$ by integrating the larger distribution wrt $f(X_2)$
        \begin{itemize}
            \item Similar to how we integrated over $W$ to get the predictive distribution in Bayesian linear regression
        \end{itemize}
    \item Consistency is automatically gained if our covariance function specifies entries in the covariance matrix
        \begin{itemize}
            \item Note that we wouldn't have consistency if we specified the entries of the "precision matrix" (inverse covariance matrix), as we would need to use all entries of the covariance matrix instead of just the $\theta$ we are looking for
        \end{itemize}
\end{itemize}

\paragraph{Bayesian linear regression as a GP}
\begin{itemize}
    \item We can view Bayesian linear regression model $f(X) = \phi(X)^W$ with prior $W \sim N(0, \Sigma_P)$ as a GP 
\begin{equation}
    \begin{aligned}
        m(X) = \phi(X)^T\mathbb{E}[W] = \phi(X)^T[0] = 0 \\
        k(X,X') = \phi(X)^T\mathbb{E}[WW^T]\phi(X') = \phi(X)^T\Sigma_P\phi(X')
    \end{aligned}
\end{equation}
    \item We will use a squared exponential (SE) covariance function, also known as the radial basis function (RBF) or Gaussian kernel
\begin{equation}
    k(f(X), f(X')) = \exp\left(-\frac{1}{2}\frac{|X - X'|^2}{\mathcal{L}^2}\right)
\end{equation}
    \begin{itemize}
        \item Covariance between outputs $f(X)$ and $f(X')$ is written as a function of inputs $X$ and $X'$ only (kernel trick)
        \item For SE, covariance is almost unity between outputs whose inputs are close together, and decays exponentially as inputs get further apart
        \item $\mathcal{L}$ is a length scale parameter that controls the rate of decay of the covariance function
    \end{itemize}
    \item It can be shown that SE corresponds to a Bayesian linear regression model with infinite basis functions
        \begin{itemize}
            \item For every positive definite covariance function $k(X,X')$, there exists a possibly infinite set of basis functions - Mercer theorem
            \item SE can also be obtained from the linear combination of infinite Gaussian-shaped basis functions
        \end{itemize}
    \item Because SE is infinitely differentiable, it produces smooth functions
\end{itemize}

\paragraph{Function evaluations to a random function}
\begin{itemize}
    \item We can choose a subset of five inputs $X_{*1}$ from our test data $X_*$ and apply a GP to get five outputs $f(X_{*1})$
    \item $f(X_{*1})$ can be described as a multivariate Gaussian distribution,e.g. in the Bayesian linear model $f(X_{*1}) \sim N(0, k(X_{*1}, X_{*1}))$ 
        \begin{itemize}
            \item Each output $f(X_{\theta*1})$ in our $f(X_{*1})$ vector is a random variable with mean $0$ and covariance with each other $K_{\theta\theta'} = k(X_{\theta*}, X_{\theta'*})$ 
        \end{itemize}
    \item There exists some random function $g(X_{*1})$ for our subsets such that $f(X_{*1}) = g(X_{*1})$
    \begin{itemize}
        \item We only know the value of $g(X_{*1})$ at the points $X_{*1}$, so $g(X_{*1}) = {X_{*1} : f(X_{*1})}$
        \item Because $g(X)$ entirely consists of random points, we can think of $g(X_{*1})$ as a random function
        \item $g(X)$ is continuous because SE guarantees consistency
    \end{itemize}
\item Thanks to consistency, if we marginalised out our subset from the entire distribution $f(X_*)$, we would recover the subset distribution $N(0, K_*(X_{*1}, X_{*1}))$ that describes our random function $g(X_{*1})$
    \item Therefore, the specification of the covariance function imples that our GP can also be seen as a distribution of o, where each sample produces a random function $g(X_*)$ that passes through the points $f(X_{*1})$
\end{itemize}


\subsubsection{Predictive distributions with noise-free observations}

\paragraph{Prior distribution over functions}
\begin{itemize}
    \item We are trying to get the best function $f$ that describes our training data $f(X)$ and test data $f(X_*)$
    \item Initially we consider the simple special case where the observations are noise free, i.e. we know $f(X)$ is the $y$ we observe
    \item Therefore, our training data $f(X)$ and test data $f(X_*)$ are jointly distributed according to the prior
\begin{equation}
    \begin{pmatrix}
        f(X) \\ f(X_*)
    \end{pmatrix} \sim N\left(
    \begin{pmatrix}
        0 \\ 0
    \end{pmatrix},
    \begin{pmatrix}
        K(X,X) & K(X,X_*) \\ K(X_*,X) & K(X_*,X_*)
    \end{pmatrix}
    \right)
\end{equation}
\end{itemize}

\paragraph{Posterior distribution over functions}
\begin{itemize}
    \item We are not interested in drawing random functions from the prior, but incorporating the knowledge that the training data provides about the function
    \item To get the posterior distribution over functions given the training data, we can condition the joint prior distribution on the training data
    \item Intuitively, like generating random functions $g$ and rejecting those that don't pass through the training data (but computationally impossible like this)
    \item Probabilistically, we need to condition our joint Gaussian prior distribution on the observations, or produce an expression for $p(f(X_*)|X_*, X, f(X))$
    \item We can substitute our prior and our data we're conditioning it on into the Gaussian multivariate conditioning identity:
\begin{equation}
    \begin{aligned}
        p(f(X_*)|X_*, X, f(X)) \sim N( \\
        [0] + [K(X_*,X)][K(X,X)]^{-1}([f(X)] - [0]), \\
        [K(X*,X*)] - [K(X_*,X)][K(X,X)]^{-1}[K(X,X_*)] \\
        )
    \end{aligned}
\end{equation}
    \begin{itemize}
        \item Note that although we condition on $X_*$, $X$, and $f(X)$, we only substitute $f(X)$
        \item $X_*$ and $X$ are known constants, but $f(X)$ is random because it is a sample from the prior
        \item We also swap $f(X_*)$ and $f(X)$ in our prior to match the conditioning identity, such that our input vector into the conditioning identity is $(f(X_*), f(X))^T$
    \end{itemize}
    \item Simplifying the last term in the mean we get our final expression for the posterior distribution
\begin{equation}
    \begin{aligned}
        p(f(X_*)|X_*, X, f(X)) \sim N( \\
        K(X,X_*)K(X,X)^{-1}f(X), \\
        K(X_*,X_*) - K(X_*,X)K(X,X)^{-1}K(X,X_*) \\
        )
    \end{aligned}
\end{equation}
\end{itemize}


\subsubsection{Predictive distributions with noisy observations}

\paragraph{Noisy observations prior}
\begin{itemize}
    \item It is typical for more realistic situations to not have the observation $f(X)$ directly as our training data, but have $y = f(X) + \epsilon$, where $\epsilon$ is a noise term
    \item Linear model assumes $\epsilon$ is additive, independent and identically distributed
\begin{equation}
    \text{Cov}(y_p, y_q) = K(X_p, X_q) + \sigma^2_n\delta_{pq}
\end{equation}
    \begin{itemize}
        \item $\delta_{pq}$ is the Kronecker delta, which returns 1 if indices ($p$, $q$) are equal and 0 otherwise
        \item $\sigma^2_n$ is the noise variance, which is a constant for all observations
        \item Note that the subscript $n$ in $\sigma^2_n$ is not the number of observations, but a reminder it's noise and not part of the prior
    \end{itemize}
    \item More generally,
\begin{equation}
    \text{Cov}(Y) = k(X,X) + \sigma^2_nI
\end{equation}
    \begin{itemize}
        \item $\sigma^2_nI$ is the noise multiplied by the correlation matrix $I$, which under independence is our identity matrix
    \end{itemize}
    \item This gives us this prior
\begin{equation}
    \begin{pmatrix}
        Y \\ f(X_*)
    \end{pmatrix} \sim N\left(
    \begin{pmatrix}
        0 \\ 0
    \end{pmatrix},
    \begin{pmatrix}
        K(X,X) + \sigma^2_nI & K(X,X_*) \\ K(X_*,X) & K(X_*,X_*)
    \end{pmatrix}
    \right)
\end{equation}
\end{itemize}

\paragraph{Noisy observations posterior}
\begin{itemize}
    \item As before, we can form a predictive distribution using the Gaussian multivariate conditioning identity
\begin{equation}
    \begin{aligned}
        p(f(X_*)|X_*, X, Y) \sim N( \\
        K(X_*,X)[K(X,X) + \sigma^2_nI]^{-1}Y, \\
        K(X_*,X_*) - K(X_*,X)[K(X,X) + \sigma^2_nI]^{-1}K(X,X_*) \\
        )
    \end{aligned}
\end{equation}
    \item Note that substituting $K(X,X') = \phi(X)^T\Sigma_p\phi(X')$ in gives us the exact same result in the weight-space view's final predictive distribution in the alternative formulation using the Sherman-Morrison identity
    \item Note that our mean is solely in terms of $Y$ and our covariance function, so it can be seen as as a linear predictor
    \item Alternatively, we can write it as a linear combination of $n$ kernel function evaluations at the training data points $X_i$ by defining $\alpha_i = K(X_i, X_i) + \sigma_n^2I)^{-1}Y_i$
\begin{equation}
    \mathbb{E}_{p(f(X_*)|X_*,X,Y)}[f(X_*)] = \sum_{i=1}^n \alpha_i k(X_*, X_i)
\end{equation}
    \item Representer theorem - we can express the posterior mean as these linear combinations despite our GP being represented as a (possibly infinite) number of basis functions
    \begin{itemize}
        \item Our prior defines a joint Gaussian distribution over all $y$ variables, one for each point in the index set $\mathcal{X}$, which could be infinite
        \item Since we are conditioning our prior on a finite number of $X_i$, we only care about the distribution of the $n$ training points and the $1$ test point we are predicting for
        \item This is the block of the joint covariance matrix we extract by marginalising
    \end{itemize}
    \item Note that our variance is independent of the targets $y$, only our inputs $X$ and $X_*$
    \item Our variance is two terms: our prior covariance $K(X_*,X_*)$, a term representing the information the observations give us about the function
    \item We can compute the predictive distribution of $Y_*$ by adding the noise term $\sigma^2_nI$ to the variance
\end{itemize}

\paragraph{Marginal likelihood}
\begin{itemize}
    \item We need some measure of how well our GP fits the data, which we can get by computing the marginal likelihood $p(Y|X)$
\begin{equation}
    p(Y|X) = \int p(Y|f,X)p(f|X) df
\end{equation}
    \item $p(f|X)$ is our prior distribution $\sim N(0, K)$
    \begin{itemize}
        \item Our mean will always be the same under our prior, but our $K(X,X')$ tells us how "wiggly" our function is
        \item The closer our $p(f|X)$ distribution is to the true complexity of the function, the higher our marginal likelihood
        \item e.g. for SE, if our data is close together $|X - X'|$ is small then our covariance $k(X,X')$ on our function distribution prior is large, so we get a high variety of functions and thus a higher probability of sampling a more complex function
        \item If we specify a lower $\mathcal{L}$, we can "artificially" get a high $k(X,X')$ 
        \item So if we supply some new data $k(X_*, X_*')$ that is far away from each other and would require a more complex function, its density in this prior smooth distribution will be low, so we get a low $p(f|X_*)$
    \end{itemize}
    \item $p(y|f,X)$ is our familar predictive distribution $p(y|f,X) \sim N(f, \sigma^2_nI)$, and represents how good a fit our function $f$ is to the training data labels $y$ and inputs $X$
    \item We can express $p(Y|X)$ as a Gaussian integral over the joint distribution of $f$ and $Y$, which yields this PDF of the marginal likelihood
\begin{equation}
    log p(Y|X) = -\frac{1}{2}Y^T(K + \sigma^2_nI)^{-1}Y - \frac{1}{2}log|K + \sigma^2_nI| - \frac{n}{2}log(2\pi)
\end{equation}
    \item Alternatively, $y = f(X) + \epsilon$, where we assume $\epsilon \sim N(0, \sigma^2_nI)$
    \item Since these are both Gaussian, we can simply add their means and variances to get $p(Y|X) = N(0, K + \sigma^2_nI)$
    \item This itself is a Gaussian that we can plug into our Gaussian PDF to get the same result
\end{itemize}

\paragraph{Practical algorithm}
\begin{enumerate}
    \item Take in inputs $X$, outputs $y$, covariance function $k$, noise level $\sigma^2_n$, and test input $X_*$
    \item Invert our $[K(X,X) + \sigma^2_nI]$ matrix needed for mean variance using Cholesky decomposition
    \begin{enumerate}
        \item $L = \text{cholesky}(K(X,X) + \sigma_n^2I)$
    \end{enumerate}
    \item Prepare the mean of our predictive distribution in linear combination form by computing the $\alpha$ vector
    \begin{enumerate}
        \item $alpha = L^T \backslash (L \backslash y)$
    \end{enumerate}
    \item Actually compute the mean
    \begin{enumerate}
        \item $mu = K(X_*, X)^T \cdot alpha$
    \end{enumerate}
    \item Prepare to compute variance by computing $v$, the form in which $L$ is used in the variance
    \begin{enumerate}
        \item $v = L \backslash K(X_*, X)^T$
    \end{enumerate}
    \item Compute the variance
    \begin{enumerate}
        \item $var = K(X_*, X_*) - v^T v$
    \end{enumerate}
    \item Compute the log marginal likelihood 
    \begin{enumerate}
        \item $\text{log} p(Y|X) = -\frac{1}{2}y^T \cdot alpha - \frac{1}{2}log|K(X,X) + \sigma^2_nI| - \frac{n}{2}log(2\pi)$
    \end{enumerate}
    \item Return the mean $mu$, variance $var$, and log marginal likelihood
\end{enumerate}


\printbibliography

\end{document}
