ection{Computational Issues}


\subsection{Matrix inversion \cite{big-data}}
Inverting the $[K(X,X) + \sigma^2_nI]$ matrix in our predictive distribution scales poorly with the number of training data points $n$, as inverting the $n \times n$ matrix $X$ that represents our training data is $O(n^3)$. Strategies to approximate the result of this inversion fall into two categories: those that produce a single approximation for the entire dataset, or those that produce several approximations that are "experts" in a particular region of the dataset and combine these local approximations to form a global approximation.

\subsection{Global approximations}

\subsubsection{Subset-of-data}
The simplest strategy is to use a subset $M$ of $X$ to reduce the cost of inversion to $O(m^3)$, where $m$ is the number of training points in $M$. Although this approach does not address the issues of matrix inversion directly,a theoretical graphon analysis proves that choosing $M$ randomly gives an accuracy of $O(log^{-1/4}m)$ for the predictive mean and variance, which produces more accurate predictions with faster runtimes than sparse approximations as $n$ increases. \cite{random-subsampling} Subset-of-data also requires no analytic assumptions about the kernel.

% using a subset of $M$ that is representative of the entire dataset by clustering the data, e.g. using a k-means algorithm, and using these cluster centroids as our subset. 

We can reduce $m$ needed to achieve the same level of accuracy with a "greedy" approach by determining the gain in likelihood from including each data point $x_i$ in X, adding the maximum gain in likelihood point to $M$ and repeating until the size of $M$ reaches $m$. However, computational savings from reducing $m$ are smaller than the cost of searching $X$ for these centroids $O(n^2m)$. Instead, we can use a "matching pursuit" approach - maintain a cache of the already precomputed kernel values, and use these to compute the gain in likelihood for each point in $X$ in $O(nm^2)$ time. \cite{matching-pursuit}

\subsubsection{Sparse kernels}
A sparse kernel is a particularly designed kernel that imposes $k(X,X') = 0$ if $|X - X'|$ is larger than some threshold $d$ to create a sparse covariance matrix. This reduces the number of calculations that need to be performed and computational complexity to $O(an^3)$, where $a$ is the proportion of non-zero entries remaining, but the kernel needs to be carefully designed to work with zeroes and ensure all entries are positive definite to satisfy completeness. TODO sparse RBF

\subsubsection{Sparse approximations}
TODO, missing background

\paragraph{Prior approximation}
\paragraph{Posterior approximation}
\paragraph{Structured sparse approximation}


\subsection{Local approximations}
TODO

\subsubsection{Naive-local-experts}

\subsubsection{Mixture-of-experts}

\subsubsection{Product-of-experts}


\subsection{Improvements}
TODO

\subsubsection{Scalability}

\subsubsection{Capability}


% \subsubsection{Extensions}
% 
% \paragraph{Scalable manifold GP}
% 
% \paragraph{Scalable deep GP}
% 
% \paragraph{Scalable online GP}
% 
% \paragraph{Scalable multi-task GP}
% 
% \paragraph{Scalable recurrent GP}
% 
% \paragraph{Scalable GP classification}

