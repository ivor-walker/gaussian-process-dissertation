\section{Computational Issues}

Inverting the $[K(X,X) + \sigma^2_nI]$ matrix in our predictive distribution scales poorly with the number of training data points $n$, as inverting the $n \times n$ matrix $X$ that represents our training data is $O(n^3)$. There exist significantly more powerful approximations of this inversion for "celerite" kernels. Strategies for any covariance function fall into two categories: those that produce a single approximation for the entire dataset, or those that produce several approximations that are "experts" in a particular region of the dataset and combine these local approximations to form a global approximation.

TODO issues with local approximations

% \subsection{Global approximations }

\subsection{Subset-of-data (SoD) \cite{big-data}}
As discussed previously, matrix approximations like the Nystrom approximation can produce kernels that violate PSD due to the irreduceable error involved. Instead of producing a low-rank approximation of the covariance matrix from a full GP, we could guarantee PSD by applying the GP to a subset $M$ of $X$ to reduce the cost of inversion to $O(m^3)$, where $m$ is the number of training points in $M$. A theoretical graphon analysis showed that choosing $M$ randomly gives an accuracy of $O(log^{-1/4}m)$ for the predictive mean and variance, which produces more accurate predictions with faster runtimes than sparse approximations as $n$ increases. \cite{random-subsampling} Subset-of-data also requires no analytic assumptions about the covariance functions.

% using a subset of $M$ that is representative of the entire dataset by clustering the data, e.g. using a k-means algorithm, and using these cluster centroids as our subset. 

We can reduce $m$ needed to achieve the same level of accuracy with a "greedy" approach by determining the gain in likelihood from including each data point $x_i$ in X, adding the maximum gain in likelihood point to $M$ and repeating until the size of $M$ reaches $m$. However, computational savings from reducing $m$ are smaller than the cost of searching $X$ for these centroids $O(n^2m)$. Instead, we can use a "matching pursuit" approach - maintain a cache of the already precomputed kernel values, and use these to compute the gain in likelihood for each point in $X$ in $O(nm^2)$ time. \cite{matching-pursuit}

\subsection{Sparse kernels \cite{big-data}}
A sparse kernel is a particularly designed kernel that imposes $k(X,X') = 0$ if $|X - X'|$ is larger than some threshold $d$ to create a sparse covariance matrix. This reduces the number of calculations that need to be performed and computational complexity to $O(an^3)$, where $a$ is the proportion of non-zero entries remaining, but the kernel needs to be carefully designed to work with zeroes and ensure all entries are PSD. TODO sparse RBF


\subsection{Sparse approximations \cite{big-data}}
SoD throws away all data not in the selected subset of real training data $m$, losing information and reducing accuracy. Sparse approximation approaches are more accurate than SoD whilst achieving the same computational cost $O(nm^2 + m^3)$, by using potentially imaginary inducing points $X_m$ that best approximate the full covariance matrix $K(X,X)$. 

Joining a new prior on $f(X_m)$ to our previous prior on $f(X)$ \ref{eq:joint_prior} gives us a new joint prior:
\begin{equation*}
    \begin{pmatrix}
        f(X)
        f(X_m) \\
    \end{pmatrix} \sim \mathcal{N} (
    \begin{pmatrix}
        0 \\
        0 \\
    \end{pmatrix},
    \begin{pmatrix}
        K(X, X) & K(X, X_m) \\
        K(X_m, X) & K(X_m, X_m)    
    \end{pmatrix}
    )
\end{equation*}
Similarly to \ref{eq:conditioning}, we can use the Gaussian conditioning identity to condition this joint prior on our inducing points $f(X_m)$ and obtain a posterior distribution over $f(X)$:
\begin{equation} \label{eq:conditioning_approx}
    \begin{aligned}
        p(f(X) | f(X_m)) \sim \mathcal{N} ( \\
        K(X, X_m) K(X_m, X_m)^{-1} f(X_m), \\
        K(X, X) - K(X, X_m) K(X_m, X_m)^{-1} K(X_m, X) \\
        )
    \end{aligned}
\end{equation}
We can show this approach is equivelant to our original priors once we marginalise out $f(X_m)$:
\begin{equation*}
    p(f(X)) = \int p(f(X) | f(X_m)) p(f(X_m)) df(X_m)
\end{equation*}
Without any approximations, it can be shown \cite{big-data} that this integral recovers our full priors on $f(X)$ \ref{jeq:joint_prior}:
\begin{equation*}
    p(f(X)) = \mathcal{N}\left(0, K(X, X)\right)
\end{equation*}
We can also assemble our predictive distribution by forming a new joint prior and a posterior with the testing data $X_*$ by substituting $X$ for $X_*$:
\begin{equation*}
    \begin{aligned}
        p(f(X_*) | f(X_m)) &\sim \mathcal{N} (
        K(X_*, X_m) K(X_m, X_m)^{-1} f(X_m),
        K(X_*, X_*) - K(X_*, X_m) K(X_m, X_m)^{-1} K(X_m, X_*)
        ) \\
    \end{aligned}
\end{equation*}
As with training data, marginalising out $X_m$ yields our original full prior on the training data $f(X_*) \sim \mathcal{N}(0, K(X_*, X_*))$.

We have introduced a reversible first step into the assembly of the predictive distribution that determines our priors on $f(X)$ and $f(X_*)$ with no additional computational cost. By introducing approximations into this new step, we assemble $f(X)$ priors that are faster to work with than the full priors. These approximations are divided into two categories: prior approximations, which change $p(f(X) | f(X_m))$, and posterior approximations, which change how the final predictive disitribution is assembled given $p(f(X))$.

\subsubsection{Prior approximations}

\paragraph{Subset-of-Regression (SoR) \cite{sor}}
SoR sets the variance of \ref{eq:conditioning_approx} to zero:
\begin{equation*}
    p_{\text{SoR}}(f(X_*) | f(X_m)) \sim \mathcal{N} (
    K(X_*, X_m) K(X_m, X_m)^{-1} f(X_m),
    0
    )
\end{equation*}
Producing a marginal prior:
\begin{equation*}
    p_{\text{SoR}}(f(X_*)) = \int p_{\text{SoR}}(f(X_*) | f(X_m)) p(f(X_m)) df(X_m)
\end{equation*}
Evaluating this integral and representing the result as a Gaussian distribution:
\begin{equation*}
    p_{\text{SoR}}(f(X_*)) = \mathcal{N}(
    0, 
    K(X_*, X_m) K(X_m, X_m)^{-1} K(X_m, X_*)
    )
\end{equation*}

Inverting the $m$-rank $K(X_m, X_M)^{-1}$ costs $O(m^3)$, and multiplying it by the $n$-rank $K(X_*, X_m)$ and $K(X_m, X_*)$ costs $O(n^2m)$, sped up using SMW \ref{eq:A_phi_inverse_phi_X} to yield $O(nm^2)$, for a total computational complexity of $O(nm^2 + m^3)$.

This achieves the same computational cost as the Nystrom approximation but does so in a probabilistic framework that guarantees PSD. However, setting our prior's variance to zero produces a predictive distribution that severely underestimates uncertainty, producing GPs that are too confident far from $X_m$.

\paragraph{Fully independent training conditional (FITC) \cite{fitc}}
FITC assumes that the new data $f(X_*)$ is independent given the inducing points $f(X_m)$. Formally:
\begin{equation*}
    p(f(X_*) | f(X_m)) = \prod_{i=1}^n p(f(x_{i*}) | f(X_m))
\end{equation*}
TODO more motivation on prior yields. Our new prior distribution is:
\begin{equation*}
    p_{\text{FITC}}(f(X_*) | f(X_m)) = \mathcal{N}(
    f(X_*) | K(x_i, X_m) K(X_m, X_m)^{-1} f(X_m),
    \text{diag}[V_{nn}] 
    )
\end{equation*}
$V_{nn} = K(X_*, X_*) - K(X_*, X_m) K(X_m, X_m)^{-1} K(X_m, X_*)$. Our new predictive distribution becomes:
\begin{equation*}
    \begin{aligned}
        p_{\text{FITC}}(f(X_*)) \sim \mathcal{N}(
        k(X_*, X_m) \lambda^{-1} K(X_m, X_m) f(X_m),
        K(X_*, X_*) - K(X_*, X_*) - K(X_*, X_m) K(X_m, X_m)^{-1} K(X_m, X_*) + K(X_*, X_m) \lambda^{-1} K(X_m, X_*) \\
        )
    \end{aligned}
\end{equation*}
$\lambda = K(X_m, X_m) + K(X_m, X_*) \text{diag}[V_{nn}]^{-1} K(X_*, X_m)$. 

There are three matrices being assembled here: the at-most rank $n$ diag[$V$], the rank-$m$ $\lambda$ and the final predictive variance of rank $m$. $V$ is similar to SoR and costs $O(mn^2)$, whilst getting its diagonal costs $O(nm)$, and assembling $\lambda$ and the predictive variance is also $O(nm^2)$. We need to invert $\lambda$, $K(X_m, X_m)$ and $\text{diag}[V_{nn}]$ - the first two are rank $M$, and inverting a diagonal only costs $O(n)$. These operations reduce to an overall complexity of $O(nm^2 + m^3)$.

FITC costs the same complexity as SoR but achieves a more accurate approximation for variance. Additionally, Bauer et. al. \cite{fitc-heteroskedasticity} found that the diagonal correlation $\text{diag}[V_{nn}]$ represents the posterior variances of $f(X_*)$ given $f(X_m)$ enables predictive variances to grow in regions far from inducing points, capturing a form of heteroskedasticity.


\subsubsection{TODO Posterior approximations}
Titsias \cite{vfe} found that approximate priors (e.g. SoR, FITC) that produce marginal likelihoods that are far from the marginal likelihood produced by the full prior \ref{eq:log_marginal_likelihood} degrade the accuracy of practical sub-routines that are dependent on maximising the marginal likelihood, such as the choosing kernel hyperparameters (e.g. $\gamma$ in $\gamma$-exponential kernels) or the inducing points $X_m$. He proposed retaining the exact prior to keep the exact marginal likelihood, and instead approximate the final predictive distribution.

\paragraph{Variational free energy (VFE) \cite{vfe}}

\paragraph{Stochastic varational GP (SVGP) \cite{svgp}}

\subsubsection{TODO Structured sparse approximation}

\subsubsection{TODO Selecting inducing points}


% \subsection{TODO Local approximations \cite{big-data}}
% 
% \subsubsection{Naive-local-experts}
% 
% \subsubsection{Mixture-of-experts}
% 
% \subsubsection{Product-of-experts}


\subsection{TODO Approximations for Celerite kernels \cite{foreman-mackay}}


% \subsubsection{Extensions}
% 
% \paragraph{Scalable manifold GP}
% 
% \paragraph{Scalable deep GP}
% 
% \paragraph{Scalable online GP}
% 
% \paragraph{Scalable multi-task GP}
% 
% \paragraph{Scalable recurrent GP}
